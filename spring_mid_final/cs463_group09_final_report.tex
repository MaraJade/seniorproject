\documentclass[draftclsnofoot,10pt,onecolumn]{IEEEtran} %!PN

\usepackage[margin=0.5cm]{caption}
\usepackage{lipsum}

\usepackage{longtable}
\usepackage{graphicx}   
\usepackage[export]{adjustbox} 
\usepackage{epstopdf}

\usepackage{pdfpages}

\usepackage{caption}

\usepackage{amssymb}                                         
\usepackage{amsmath}                                         
\usepackage{amsthm}                                          

\usepackage{alltt}                                           
\usepackage{float}
\usepackage{color}

\usepackage{hyperref}
\usepackage{url}

\usepackage{array}

\usepackage{balance}
\usepackage[TABBOTCAP, tight]{subfigure}
\usepackage{enumitem}

\newcommand{\ignore}[2]{\hspace{0in}#2} %Used for inline comments
\newcommand{\tab}{\hspace*{2em}} %For tabbing

\renewcommand{\labelenumii}{\theenumii}
\renewcommand{\theenumii}{\theenumi.\arabic{enumii}.}

\usepackage{pstricks, pst-node}

\usepackage{geometry}
%\usepackage{graphicx}
\geometry{textheight=10in, textwidth=7.5in, margin=0.75in}

\usepackage{listings}

\definecolor{mygreen}{rgb}{0,0.6,0}
\definecolor{mygray}{rgb}{0.5,0.5,0.5}
\definecolor{mymauve}{rgb}{0.58,0,0.82}
\linespread{1}

\lstset{ %
  basicstyle=\ttfamily,            % the size of the fonts that are used for the code
  breakatwhitespace=false,         % sets if automatic breaks should only happen at whitespace
  breaklines=true,                 % sets automatic line breaking
  captionpos=b,                    % sets the caption-position to bottom
  commentstyle=\color{mygreen},    % comment style
  escapeinside={\%*}{*)},          % if you want to add LaTeX within your code
  extendedchars=true,              % lets you use non-ASCII characters; for 8-bits encodings only, does not work with UTF-8
  keepspaces=true,                 % keeps spaces in text, useful for keeping indentation of code (possibly needs columns=flexible)
  keywordstyle=\color{blue},       % keyword style
  %numbers=left,                    % where to put the line-numbers; possible values are (none, left, right)
  %numbersep=10pt,                  % how far the line-numbers are from the code
  %numberstyle=\tiny\color{mygray}, % the style that is used for the line-numbers
  rulecolor=\color{black},         % if not set, the frame-color may be changed on line-breaks within not-black text (e.g. comments (green here))
  showspaces=false,                % show spaces everywhere adding particular underscores; it overrides 'showstringspaces'
  showstringspaces=false,          % underline spaces within strings only
  showtabs=false,                  % show tabs within strings adding particular underscores
  stepnumber=1,                    % the step between two line-numbers. If it's 1, each line will be numbered
  stringstyle=\color{mymauve},     % string literal style
  tabsize=8,                       % sets default tabsize to 8 spaces
  %title=\lstname                  % show the filename of files included with \lstinputlisting; also try caption instead of title
}

\lstdefinelanguage{JavaScript}{
  morekeywords={var, function},
  morecomment=[s]{/*}{*/},%
  morecomment=[l]//,%
  morestring=[b]",%
  morestring=[b]'%
}

\usepackage{hyperref}

\def\BibTeX{{\rm B\kern-.05em{\sc i\kern-.025em b}\kern-.08em
    T\kern-.1667em\lower.7ex\hbox{E}\kern-.125emX}}
    
%\usepackage{floatrow}
\usepackage{lipsum}

    
\renewcommand{\lstlistingname}{Code Snippet:}


\begin{document}
% Hide page numbers
\pagenumbering{gobble}

%\title{Using the Style File IEEEtran.sty} 
\title{Tools for Supporting Community Growth in Open Source \\ {\large CS463: Final Report Spring 2016}}

\author{Bruntmyer J. Author, OSU, Goossens M. Author, OSU, Nguyen H. Author, OSU}

%\markboth{Tools for Supporting Community Growth in Open Source}
%{Murray and Balemi: Using the style file IEEEtran.sty} %!PN
%{Murray and Balemi: Using the Document Class IEEEtran.cls} %!PN


\maketitle
%\thispagestyle{plain}\pagestyle{plain}
\begin{abstract}
For the past six months our group has been working on a project that is creating tools that
gives users the ability to look for open source community leaders that are
hosting events. These tools will allow
users to have the opportunity to find these events in order to become a
contributor to an open source project. This is done in the form of a website that
will have features for finding certain events dealing with open source projects
so that it can be easily accessible by people with a passion for wanting to
contribute to projects. Throughout this document, we look at what this team has
accomplished for each of the requirements that have been laid out, discussing problems
that have halted our progression through the project, and how we changed our timeline.
Also included are important images of the user interface we have decided
to use, along with pieces of code that we have completed. By the end of this document, 
you will get a complete picture of how we
reached our version 1.0 release.
\end{abstract}

\newpage

\pagenumbering{arabic}

\section{INTRODUCTION TO THE PROJECT}

The Community Driven Development project is sponsored by the Apache Software Foundation under Ross Gardler, Director and President of the company. The project was designed and prototyped by Gardler before being proposed to the senior software project class for Oregon State University. Development for the project hopes to achieve more involvement in terms of helping building upon the Open Source Community and helping it grow and gain more traffic. The prototype was regularly used by Gardler and many of the faculty from Apache and is continued to gain interest to the public. The importance of development of the project will promote more growth and allow more tools for users in the community to find and locate more events and get more involved in open source community groups and projects. The students assigned to work on this project were Justin Bruntmyer, Megan Goossens, and Hai Nguyen. Each member took upon equal and separate roles for the project. No specific roles were assigned as each member took part in doing part of the work for every task. Ross Gardler provided the initial prototype while the project was improved and implemented with the requested features as the term progressed.

\newpage

\section{REQUIREMENTS DOCUMENT}
\subsection{Introduction to the Problem}
Open source projects are known for the code that is developed from them however that is not what causes these projects to either live or die. Though it is true that an open source group can survive without a viable community in most cases there is a class of open source projects that lives or dies on the strength of its community such as the Apache Software Foundation projects. Existing tools are in place for tracking activity within online communities however these tools focus on identifying customers to the open source projects. The problem that is being focused on is that new tools need to be made to identify community leaders so that community builders know who to engage with.

\subsection{Project Description}

The goal of this project is to build connections, collaborations, and identifications between community members and leaders in open source projects. This entails contributing to Apache Software Foundation’s existing software platform to create tools that identify and support community leaders and members. Better tools will help identify potential community leaders which will then allow community builders to engage with and support those leaders. Over the course of the project, consistent updates check what tools were used as intended and which ones could be improved and expanded. Evaluation of the tools implemented is done in order to measure its significance and effectiveness.

The solution to this problem should be able to identify community leaders to people with a desire to improve the strength of a community around a specific project. This solution includes being able to observe Apache Software Foundation community activity on publicly available information such as meetups.com in order to identify those potential community leaders. The solution is to provide enough information to community builders to contact and engage with the leaders so that they can begin their own contributions to the projects.

\subsection{Requirement}

The end goal of our project is to create a tool for identifying community leaders for others to see and given the chance to get involved with by having a web page that will let users go through and view all of this data in one location. This data will be gathered from an algorithm that reaches out to meetups.com and use key terms to find community leaders. As a stretch goal we would like to have this algorithm gain data from other social media sites such as Facebook and Twitter.

We will be starting with a prototype that has been implemented which means we will need to start by fixing what doesn't work and then adding key features that we believe the project needs. In a final delivery there will be a patched version of the prototype with tool fixes along with new features to improve usability. 

\newpage

\newcolumntype{C}[1]{>{\centering\let\newline\\\arraybackslash\hspace{0pt}}m{#1}}

\begin{center}
\begin{longtable}{ | C{1cm} | C{7cm}| C{6cm} | C{2cm} |} 
\hline
REQ\# & Requirement & Priority & Expected Completion \\ 
\hline
1 & Gain access to source code of the Community Developments page & HIGH & 11/30/2015\\
\hline
2 & Fix the "People" page where the list of community leaders are shown & HIGH & 01/11/2016
\\
\hline
3 & Tweet at a person listed i the database & LOW & 03/14/2016
\\
\hline
4 & Add user accounts to the application and track when a user has tweeted an event & MEDIUM & 02/15/2016
\\
\hline
5 & List tweets about events and/or people via the app & MEDIUM & 02/22/2016
\\
\hline
6 & List tweets about events and/or people from twitter, but not via the app & MEDIUM & 02/29/2016
\\
\hline
7 & Export a list of people with information & LOW & 03/07/2016
\\
\hline
8 & Improve hashtag searching of application for better results on relevant events & HIGH & 02/08/2016*
\\
\hline
9 & Alpha Release &  & 02/08/2016
\\
\hline
10 & Improve the visuals of the tool looks as a whole & HIGH & 02/01/2016
\\
\hline
11 & Beta Release &  & 03/14/2016
\\ 
\hline
12 & Implement a system of finding events nearby a location entered or within radius of the user & HIGH & 02/08/2016*
\\
\hline
13 & Add feature to generate a profile for community developers to have contacting information easily view-able & HIGH & 01/25/2016
\\
\hline
14 & 1.0 Release &  & 05/20/2016
\\
\hline
\end{longtable}
*We expect this task to be more difficult, so we will work on the task in parallel to other tasks during that time.
\end{center}

\subsection{Alpha/Beta/1.0 Releases}
We plan to release the Alpha version of the project during week 6 of winter term where we will have multiple fixes complete for the tools that are already implemented in to the project that we are starting with. This will mean that the Alpha version will be a major patch on what is created already in order to improve the functionality of the current implementation. We plan to release the Beta version of the project during finals week of winter term where we will implement new tools for community leaders to manage their posts along with some other useful tools. Here we plan to have almost all of the functionality ready. Lastly version 1.0 will be released on May 16, 2016 which is the Monday prior to the engineering expo. This version will be our final release and should be bug free in the implementation of our project.

\subsection{Specifications of Requirements}
\begin{enumerate}
\item Gain access to source code of the Community Developments page

\begin{enumerate}
\item Gain access to the source code for the prototype to begin working.
\item This will be given by client as we will be collaborating with him on testing the functionality of the prototype.
\end{enumerate}

\item Fix the “People” page where the list of community leaders are shown
\begin{enumerate}
\item Currently takes a large amount of time in order for the "people" page to load as it shows the current people who are identified as community leaders.
\item Figure out why the page takes such a long time to load. Could be too much data and the algorithm for pulling up this page needs to be changed.
\item Need to debug the causes of crashes by some browsers.
\end{enumerate}

\item Tweet at a person listed in the database.
\begin{enumerate}
\item There will be a button that a user can select to send a tweet to a person directly.
\item This will be sending the user to the twitter API in order for them to construct and send a tweet.
\end{enumerate}

\item Add user accounts to the application and track when a user has tweeted an event
\begin{enumerate}
\item This is so they don't inadvertently do it twice
\item Track date of tweet, tweet text and tweet ID
\item This enables the prototype to branch out further than meetups.com
\item Allows user to see what tweets they have tweeted about
\end{enumerate}

\item List tweets about events and/or people via the application
\begin{enumerate}
\item This should work for the current user and others
\item Allows retweet or sending as new tweet based on existing one
\item Have a button that will list the tweets that were tweeted via the application.
\end{enumerate}

\item Export a list of people and email addresses
\begin{enumerate}
\item The emails corresponding with the people will be exported as well
\item Will be in JSON, CSV and test format for easy consumption in other applications
\end{enumerate}

\item Improve hashtag searching of application to improve finding relevant events
\begin{enumerate}
\item The searching algorithm is currently returning data from meetups.com with searching for tag "apache" however this is not working correctly as it is pulling information that has nothing to do with open source projects and is very random.
\item The searching algorithm will need to be fixed so that it find the correct tags from meetups.com and pull the accurate data.
\item The stretch goal here is the be able to use this algorithm on multiple social media networks.
\end{enumerate}

\item Alpha Release
\begin{enumerate}
\item Alpha to be released the 6th week of winter term.
\item This version will have some fixes for the improvement on algorithm, mark event action, mark group action, import members action, import meetups action, and the people action.
\item If not completely fixed/implanted the project will at least handle them by throwing a message to the user saying this tool is under construction.
\end{enumerate}

\item Improve the visuals of the tool looks as a whole
\begin{enumerate}
\item Make the user interface look sharper and more appealing
\item Have the events and people information that is displayed when a user decides to look into more details about a person or even be more appealing as well.
\end{enumerate}

\item Beta Release
\begin{enumerate}
\item The Beta of the project will be released on finals week of winter term.
\item In the Beta the fixes from the fist Alpha should all be fixed correctly, and all of the tools that we wanted to add should be implemented but not necessary working correctly, at least has some functionality.
\end{enumerate}

\item Implement system of improved sorting of finding events by nearby location within radius of the user
\begin{enumerate}
\item The program will be able to get the location from the user and based on this location the nearest community leader events will be shown to the user.
\end{enumerate}

\item Add feature to generate a profile for community developers to have contacting information easily viewable
\begin{enumerate}
\item Based on the data collected from meetups.com to identify the community leader not only will the events be posted but the community leader will have a section where the user can easily view the contact information pulled from meetups.com. A link with their name.
\item This is not a collaboration communication tool but just a way to view the contact information.
\end{enumerate}

\item 1.0 Release.
\begin{enumerate}
\item This will be the final release where all of our implementations should be working correctly with no bugs.
\item This will be released on May 16, 2016 which is the Monday prior to the Expo.
\end{enumerate}
\end{enumerate}

\subsection{Risk Assessment}
\begin{enumerate}
\item Miscommunication on can happen in between the Community Development team currently working on the project.
\item Security issues, unwanted changes to source code from unauthorized individuals
\item Algorithms created cause even slower computation time for certain aspects of the website
\item If certain implementation is added and is miscommunicated to a function that was not originally discussed
\item Pulling data from meetups could be corrupted
\item Code does not match with already implemented coding standards on the project
\end{enumerate}

\subsection{Stretch Goals}
\begin{enumerate}
\item Create a notification system to alert users of important events that the user may be interested in.
\item Create profile system for users and community developers
\item Create a notification system to alert users of important events that the user may be interested in
\item The project will have a tool that will allow a user to enter their email and they will be notified when a project event is added that they may be interested in.
\begin{enumerate}
\item This could also be implemented to send emails to users that sign up for them to a certain community leader for a certain project rather than trying to figure out what they might be interested in.
\end{enumerate}

\item Fix "Import Members” to allow importing of members
\item Allow community developers to edit their own events for specific messages
\item Create profile system for users and community developers
\end{enumerate}
\subsection{Time Table Gant Chart}
The gantt chart below shows the timeline we have set for our group to meet the requirements proposed in this document. We will be trying to fulfill this timeline as much as possible in a timely manner as it gives us a great path towards completing this project.

See gantt chart on next page.

\newpage

%\includepdf[pages={-}, pagecommand={}]{Requirements_Document1.pdf}


\includepdf[pages={8, 9}, pagecommand={}]{Requirements_Document2.pdf}

\section{Changes to the Requirements}
After continuous work on this project, the perspective on what
this community development tool wants to accomplish became clear. After taking
more time to read and understand the code that was previously written for the
prototype, we can see clearly how the data is organized and connected
using Django and its tools. First and foremost, the purpose of the community
development tool remains the same. The purpose is to gather information from
meetup.com which is the website we are pulling information about events and people from.
Next we parse that information into a list of upcoming events related to
Apache and open source projects so that developers in the open source community
have an easy way to access an environment where they can hope to participate in those
events. The ultimate goal of this project is to create a set of tools that eager developers
can use to find events, view community leader profiles, and get involved.\\

\newcolumntype{C}[1]{>{\centering\let\newline\\\arraybackslash\hspace{0pt}}m{#1}}

\begin{center}
\begin{longtable}{ | C{1cm} | C{7cm}| C{6cm} | C{2cm} |} 
\hline
REQ\# & Requirement & What Happened To It & Comments \\ 
\hline
1 & Fix the "People" page where the list of people are shown from groups & The
    “People” page currently takes all of the people in the database and lists them
    onto the page. Normally, if the prototype is hosted on a local machine and the
    database is relatively small, then the page loads fine in a minimal amount of
    time. The issue is nested in the actual hosted site by Apache where hundreds of
    thousands people are imported into the database daily and dramatically slowing
    down the loading time of the page.  With our current progression of the project,
    we have not made significant progress into improving the loading time of the
    page. We use our own local host to import a small amount of members at a time
    and that requirement is set to be worked on shortly for Beta implementation. The
    guideline for working towards accomplishing this requirement is to limit the
    amount of people loaded at a time onto the page. For Beta, we have rearranged
    where the table is generated for the people page in the function within
    views.py. This change specifically was introduced because a bug was found where
    when the table is generated, then if the amount of people were too many, then
    the table would crash and not build. With the rearrangement, now the table does
    not break through a large build and now tends to load faster. Unfortunately, the
    implementation of the fix for the People page is local. We have yet to test it
    on the host that Apache is using to run the prototype currently. We predict that
    the fix will work but it will need to be approved and patched into the live site
    for clarification. & blah\\ 
\hline
2 & Tweet at a person listed in database & Each person who is imported into the
    application is generated their own profile page based off of their Meetups ID.
    Information from Meetups about the person’s profile is also parsed in the
    community development tool. Those profiles include displaying the twitter handle
    of the person. This was done by changing the Meetup API request so that we could
    get the correct information and then store that information in our database.
    This can be seen in code snippet 1 located below with the URL shown along with
    the 'if' statements to locate the twitter handle. This allows the user to get in
    contact with the person in a profile. Above the twitter handle is a button that
    has the Twitter symbol which allows the user to click and send a tweet at the
    person via Hoot-suite. The hoot-suite app is given the twitter handle of the
    person the user wants to tweet at and the URL of the persons page on our
    application for reference. The user signs in to compose the tweet and sends it
    under their Twitter account. The code snippet 2 located below shows the HTML
    encoding of the button used to create the Hootsuite connection and shows the
    retrieval of the twitter handle form the database with 'person.service'. Not all
    users have a Twitter handle registered with Meetups thus the tweet button does
    not have any use. To handle this case the tweet at button only appears on
    profiles of imported people that have a registered Twitter handle. & blah\\ 
\hline
3 & Add user accounts to the application and track when a user has tweeted an
    event & The purpose of adding user accounts to the applications is to be able to
    track when users tweeting about events or people. With the completion of this
    requirement, the account creation is working along with being able to sign in
    successfully with a confirmation of signing in by displaying a welcome message
    along with the user's username. There is also a login and logout button located
    in the top right section of the website on all pages allowing the user to login
    or logout at anytime. Everything is also backended with features of the website
    only existing if the user has an authorized account. This means that action
    functions within the application which include importing meetups, importing
    members, marking events as not applicable, and marking groups as not applicable
    to be behind being signed in.  This means if you are not logged in with the
    authorized account, you can't perform these actions. Note that the accounts
    created are allowed access to these features, but do not have access to the
    administrative page that deals with the Django database. & blah\\ 
\hline
4 & List tweets about events and/or people via the app &  We were unable to
    track the precise tweets made from our application, but we have found an
    alternative that mostly works. As shown in the code snippet, the website makes a
    call to Twitter's search API, requesting all tweets that contain a certain
    hashtag as well as the hashtag \#Meetup. The hashtags are the same as the ones
    used to get events from meetup.com. Once it has the tweets, it uses the id from
    each one to make another call to Twitter's OEmbed API, which sends back HTML
    that is used in the page's template to present embedded tweets to users. This
    still pulls a few tweets that are unrelated, but bit of filtering would work.
    Unfortunately this would be difficult, and is outside the scope of our project.
    & blah\\ 
\hline
5 & List tweets about events and/or people from twitter, but not via the app &
    The website now has a tweet parser. As shown in the code snippet, it sends a
    call to Twitter's search API, requesting all tweets with a certain hashtag. The
    hashtags are the same as the ones used to get events from meetup.com. Once it
    has the tweets, it uses the id from each one to make another call to Twitter's
    OEmbed API, which sends back HTML that is used in the page's template to present
    embedded tweets to users.  The search results currently contain all instances of
    the hashtag requested, even when they are not relevant to any event, or even
    open source. The search needs refinement, however, this will be difficult, and
    is not within the scope of the project & blah\\ 
\hline
6 & Export a list of people with information & When taking on this requirement
    we quickly realized that the Meetup API would not provide email address for its
    users which was understandable. We then looked at what other information would
    be useful to extract about the people that were loaded into the database. This
    lead us to export information such as name, twitter handle, bio, Meetup ID, URL,
    country, state, and city. The export can be executed by clicking on the 'Export
    Info' button located in the top left of the people page and creates a file in a
    XLSX format which can be directly opened or saved. & blah\\ 
\hline
7 & Improve hashtag searching of application for better results on relevant
    events & The solution to this issue was to change one word in the call to the
    meetup.com API. In the API there are many different restrictions you can use to
    request events. Two of them come into play here: "text" and "topic." The text
    query searches through the content of the events, while the topic query looks at
    the topics related to the group. The original query was looking at text, which
    resulted in a lot of unrelated events. The query was fixed to use "topic," which
    has reduced unwanted events significantly. A thourough examination of events
    imported resulted in no unrelated events pulled in. & blah\\ 
\hline
8 & Implement a system of finding events nearby a location entered or within
    radius of the user & This tool creates a list of all events parsed through the
    application and displays a marker for each even onto the Google Map displayed to
    the right of the lists of events. A user can search for a specific event by
    looking through the list and then seeing where this is on the map. This feature
    also asks the web browser of the user for the geo-location of the user in order
    to place a special marker on the map showing where the user is in reference to
    the rest of the markers. The user can decide weather or not to accept giving the
    their location to the application. The map is generated through Google Maps by
    using the Google Map API calls. This map was implemented with a search bar
    allowing the user to search for a location and see what events are in that
    location. With each search the map jumps to the searched location and is given a
    200 mile radius circle to show what events are within 200 miles of the user. In
    order to get the markers of each event to show up on the map the latitude and
    longitude of each event needed to be stored and accessed by the map. This was
    done by a Meetups API call as shown in code snippet 3. Once the API call is made
    the json object is returned and parsed to obtain information on the event
    including name, longitude, and latitude. Once the information is stored it is
    accessed in the JavaScript for the Google Map which can be seen in code snippet
    4 below. This is done by looping though a list of events and gather the name,
    longitude, and latitudes in order for the markers to display. When a user hovers
    over a marker on the map the name of the event is shown.  When there are no
    events imported a message stating "No Events Available" is displayed. & blah\\ 
\hline
9 & Add feature to generate a profile for community developers to have
    contacting information easily visable & The main goal of the tool is to promote
    community development in the open source scene. What this feature tackles is a
    way to display important information about people that are already involved with
    projects to those who would like to get involved. This feature generates
    profiles for event hosts along with members of groups that have been imported by
    the user. These profiles consist of the of selected person, Twitter handle,
    location, link to Meetups profile, last activity date, group associated with,
    topics they are interested in, a biography, and a picture of themselves.  In
    order to gather the information for these we had to make another API call to
    Meetups. First, we had to adjust the current API call for importing events to
    also gather the ID of the event hosts.  Once this ID was obtained the next step
    is another API call gathering information on each event hosts while searching
    with the ID's gathered. Once this information was obtained the profiles for
    event hosts and imported group members could be displayed in their own sections
    of the application.  With users having the options to see group members and
    event hosts there are more opportunities for these users to get involved. &
    blah\\ 
\hline
10 & Improve the visuals of the tool and how it looks as a whole. & The
    application itself began as a fairly organized piece. The navigation bar
    implementation really helps the user keep track navigating between each page.
    When viewing the list of events, the events are listed in chronological order
    starting with the most recent. There is a search bar available for the user to
    type in for a certain event that they wish to view. There are other sorting
    mechanisms to view those events in another sorting order.  Our focus in this
    requirement were to mainly to improve how data is displayed. This specifically
    applies to to the event page and people profiles. Along with the addition of the
    Host objects, the generated host profiles would have a visual update as well.
    As displayed in Figure 9, the improvement of the profile page is shown with
    categories specifically labelled and displayed in concrete areas of the page. If
    the certain variable does not exist, then the user can clearly see where that
    detail would go on the page.  Topics are more clearly organized have are
    configured under a scrollable table as well.  Overall, this addition makes the
    people profile page much more organized and legible.  In addition to the updated
    people profile page, the event page is also upgraded in Figure 10. Similar to
    the profile upgrade, the event page now has categories that specifically detail
    where objects belong. This is much more particularly important where the
    description is displayed for the event. Previously, it was much more difficult
    for the user's eye to view where the venue of event would be. Now in this
    update, that portion is clearly labelled which allows the user to spend less
    time reading and to quickly see the information that they want to. & blah\\ 
\hline
\end{longtable}
\end{center}


%The design document is inserted here.
\section{Design Document}

\begin{enumerate}
\item FRONTSPIECE \\
	\begin{enumerate}
	
		\item Date of Issue and Status \\
		The design of the project was issued on October 4 th , 2015. The status of the project was already in production with a
		prototype.\\
		
		\item Issuing Organization \\
		The Apache Software Foundation originally began development of the project and issued the proposal to Oregon
		State University for continued development. This entails to contributing to Apache Software Foundation’s existing
		software platform to create tools that identify and support community leaders and members.\\
		
		\item Authorship \\
		Current developers contributing to the community tool include Ross Gardler, Justin Bruntmyer, Hai Nguyen, and
		Megan Goossens. Ross Gardler, president of Apache, is the original developer for the project.\\
		
		\item Change History \\
		
	\end{enumerate}
	
\item INTRODUCTION \\
	\begin{enumerate}
	
		\item Purpose \\
		The primary purpose of this document is to present a detailed description of the design elements of the Getting
		Connected: Tools for the Open Source Community project. This will guide the group in the design of the
		application.\\
		
		\item Scope \\
		This document will provide details on the design of the various technologies of the web application, in particular the
		Django framework, PostgreSQL, HTML user interface, and the user and group authentication system provided by Django.\\

		The user will have the ability to view the community leaders that are posting events on social media sites based on
		tags fetched by the tools. This will allow the user to see the activity and contact information of the community leader
		to potentially contact if the user is interested in contributing. The user will be able to view a summary of the event as
		well as have a link to the source of the event hosted by the community leader.\\
		
		\item Context \\
		This project will be free to access for everyone. Development and maintenance will have not cost as this project is
		open source. Future development plans will be based on the features that do not make it in the 1.0 release of the
		application. There are several stretch goals that will potentially be incorporated. These features are not covered in
		this document.\\
		
		\item Summary \\
		This document will go over the design for the aspects of the project including web framework, database design, user
		interface, and authentication system. \\
	\end{enumerate}

\item REFERENCES \\
	\begin{enumerate}
	
		\item IEEE Standard for Information Technology--Systems Design--Software Design Descriptions, 1st ed. IEEE, 2009. \\
		
		\item Comdev1-us-west.apache.org, 'Community Development: Events', 2015. [Online]. Available: http://comdev1-us-
		west.apache.org/events/. [Accessed: 03- Dec- 2015]. \\
		
		\item Djangoproject.com, 'The Web framework for perfectionists with deadlines | Django', 2015. [Online]. Available:
		https://www.djangoproject.com/. [Accessed: 03- Dec- 2015]. \\
		
		\item W3.org, 'HTML5', 2015. [Online]. Available: http://www.w3.org/TR/html/. [Accessed: 03- Dec- 2015]. \\
		
		\item Postgresql.org, 'PostgreSQL: The world's most advanced open source database', 2015. [Online]. Available:
		http://www.postgresql.org/. [Accessed: 03- Dec- 2015]. \\
		
	\end{enumerate}

\item GLOSSARY

\begin{longtable}[1]{ | C{7cm} | C{7cm} |} 
\hline
Term & Definition \\ 
\hline
API & Application Programming Interface\\
\hline
Query & Request sent to a database
\\
\hline
Query Parameter & Argument sent to a database to refine a search
\\
\hline
Community Leader & Someone who is in charge of a project or organization
\\
\hline
Tags &  Keywords - e.g. "apache," "open source," etc.
\\
\hline
HTML & HyperText Markup Language
\\
\hline
Django & Web framework
\\
\hline
PostgreSQL & Database management system
\\
\hline
Meetup & Meetup.com
\\
\hline
\end{longtable}

\item BODY \\
	\begin{enumerate}
		\item Identified Stakeholders and Design Concerns \\
		For the stakeholders, the tools support and help the Apache Software Foundation. Specifically, Ross Gardler is the
		main stakeholder that is part of the design.\\
		
		Concerns regarding the design of the project include completions of certain implementations with the requirements
		listed. Troubleshooting problems or possible encounters that may appear during implementation are listed as such.\\
		
		\item Algorithm Viewpoint \\
			\begin{enumerate}
				\item Fix the "People" page where profiles are not implemented \\
				As a high priority, fixing any type of design bug that was already implemented in the prototype is
				important. The page listed currently takes a very long time as a link from extracting all data from the
				PostgreSQL database causes difficulty in the time frame. Possible ways to approach in this solution
				include:
				\begin{itemize}
					\item Taking a look at how the data is sorted
					\item Looking at the users list and what is included in "People"
					\item Changing the algorithm in the time frame to extract the data.\\
				\end{itemize}
				
				\item Design View %pictuee goes here!!!!!!!!!!!!!!!!!!!!!!!!!!!!!!!!!!!!!!!!!!!!!!!!!!!!!!!!!!!!!! \\
				Our fixing the people page viewpoint shows how the application will be interacting with the Django
				framework along with the database. We know that the button on the webpage will be an HTML button that
				will be selected by the user. Once selected the Django framework will be activated to query the database
				for the current list of community leaders. Once the database finishes the query it will send the list back to
				the HTML page to be displayed for the user.\\
				
				\item Design Rationale \\
				A look at how the various steps for loading the people page is necessary due to the fact that it is not
				obvious as to why the page is not loading properly. As there are several possible sources for the slowness,
				all of them must be examined to be able to determine the cause or causes. Once the problem has been
				pinpointed, the next step is to work on optimizing it, whether that involves changing how the data is stored,
				changing the query that fetches the data, or changing the algorithm that processes and presents the fetched
				data.\\
				
				\item Improve searching algorithm of tool for more relevant events\\
				The way the events are listed and how relevant they are to the user is important. With the prototype, events
				currently being extracted have the ability to not even be related to open source or programming itself.
				Implementing a new searching algorithm will be difficult and could cause a lot of troubleshooting listed
				below:
				\begin{itemize}
					\item Improving methods of recognizing relevant tags and keywords
					\item Removing any event that has no relation to open source projects
					\item Negative search words
					\item Analyse existing non-applicable groups \\
				\end{itemize}
				
				\item Design View \\ %picture goes here!!!!!!!!!!!!!!!!!!!!!!!!!!!!!!!!!!!!!!!!!!!!!!!!!!!!!!!!!!
				Our improvement with the searching algorithm viewpoint is shown below and mainly shows how the
			    application will work with an efficient algorithm for searching the events. With the correct algorithm the
				Django framework will periodically query the meetups public posts searching for tags that have to do with
				the open source projects community. Once it returns those events they will be sent to the database and be
				stored for the HTML request will be sent to the
				Django framework where the query will be sent to the database to show the list of events on the application
				via HTML.\\
						
				\item Design Rationale \\
				There are two parts to this viewpoint: The fetching of events from Meetup, and showing the events to the
				user. Getting the events from Meetup will be a regular, scheduled event that will happen at appropriate
				intervals. Constant fetching from Meetup, or fetching every time the events page is loaded, is impractical
				and would slow the website down considerably. Once the most recent set of events has been stored in the
				database, it is a relatively simple matter to fetch them from the database when the events page is loaded. \\
				
			\end{enumerate}
			
		\item Context Viewpoint \\
			\begin{enumerate}
				\item Tweet at a person listed in the database \\
				Implementing another social media aspect other than just meetups.com is important for the project. This
				tool looks to connect every various source of data and also be able to give the user a tool to communicate
				with either a community leader, or a current user working on the project. Currently the requirement priority
				is listed as low, but possible issues for this requirement include:
				\begin{itemize}
					\item The time frame of learning how to implement different social media
					\item Implementing the feature itself integrating with already registered twitter accounts \\
				\end{itemize}
				
				\item Design Viewpoint %picture goes here !!!!!!!!!!!!!!!!!!!!!!!!!!!!!!!!!!!!!!!!!!!!!!!!!!!!!!
				Our viewpoint for the ability to tweet at a community leader is shown below and has details regarding
				communication between the application, Django framework and the database. There will be an HTML
				button that can be pressed by the user and when this is pressed the Django framework will query the
				database for that community leaders twitter user name which will then be sent back to the Django
				framework. From here the Django framework will create the necessary query parameters that will be sent to
				the Twitter API. Once this is done the user should see a new page with the Twitter API already having the
				username ready and they can just enter the tweet they would like to send. \\
				
				\item Design Rationale \\
				Twitter has a way to create tweets for you with the information you wish to be contained in them. Using
				their API, a pre-constructed tweet can be constructed. Every person and group's Twitter handles will bestored in the 						database, so a query will need to be made to retrieve that so that it can be added to the tweet,
				along with whatever default message is decided upon. \\
			\end{enumerate}
			
		\item Information Viewpoint \\
			\begin{enumerate}
				\item Add user accounts and track when a user has tweeted an event \\
				In hopes to increase social media feed, the tool looks to integrate movement with community leaders and
				help users consistently track their activity. As a medium priority, the most concerning implementation is to
				correctly link the account with the correct user and to distinguish their own Twitter accounts with the
				correct user account on the website. (Hoot Suite). \\
				
				\item Design View \\ %picture goes here!!!!!!!!!!!!!!!!!!!!!!!!!!!!!!!!!!!!!!!!!!!!!!!!!!!!!!!!!!!!
				Our viewpoint for creating user accounts is shown below with the communication between the application,
				Django framework, and the database. Here the user will select the create account HTML button which will
				begin the actions of the Django authentication system for creating accounts. This will bring a form to the
				application and present using HTML to the user who will enter in the required fields. Once submitted the
				Django authentication system will check the fields and then send it to the database for account creation.
				Once this is successful the application will display a message stating the account has been created
				successfully. \\
				
				\item Design Rationale \\
				As the application is build in Django, we will be using Django's authentication system for user accounts. It
				is the best choice for this project because it is simple and already implemented within the Django platform. \\
			\end{enumerate}
			
		\item Interface Viewpoint \\
			\begin{enumerate}
				\item Improve viewing method to examine the event itself \\
				Improving the way users view the website is important to make it look more pristine and clean. All
				information should be clearly sectioned and any relevant information that the user would like to know
				should be put in a convenient viewing form.
				\begin{itemize}
					\item Categorizing certain data could turn to an issue in organizing the data
					\item Some data will be missing and handling that missing information in the event should still give the
					user enough information to look for it \\
				\end{itemize}
				
				\item Design View \\ %picture goes here!!!!!!!!!!!!!!!!!!!!!!!!!!!!!!!!!!!!!!!!!!!!!!!!!!!!!!!!!!!!!!
				Our viewpoint for the improve viewing method requirement that we will be implementing is shown below.
				This shows the layout of how the information should be spread out through the website. This is a basic look
				at how each location will store data for the user to easily navigate through in order to find what they are
				searching for. There are also a couple of links that are described for the tweet at someone or tweet at an
				event. \\
				
				\item Design Rationale \\
				The goal of this viewpoint is to make it as easy and intuitive as possible to view and interact with people
				and events. It is important that the relevant data be shown in such a way that makes it easy to find what the
				user needs, as well as easy access to the original information. \\
			\end{enumerate}
			
		\item Resource Viewpoint \\
			\begin{enumerate}
				\item Implement system of improved sorting of finding events by nearby location \\
				Listed as one the more difficult requirements to complete, the task is given a large amount of time to
				complete in order to get it fully functional. It is important and more relevant for the user to view events that
				can easily be travelled to by their location. Developing a way for the application to view and sort those
				locations could cause some concerns.
				\begin{itemize}
					\item Creating different approaches in sorting locations of the events currently listed and detecting the
					location of the user
					\item Creating a database of all locations and their immediate distance from each other \\
				\end{itemize}
				
				\item Design View \\ %picture goes here!!!!!!!!!!!!!!!!!!!!!!!!!!!!!!!!!!!!!!!!!!!!!!!!!!!!!!!!!!!!!!!!
				Our viewpoint for the location nearest the user functionality is shown below. We plan to make use of how
				most internet browsers have a location setting which can tell where a user is accessing the website from.
				From this information we will query the database and look for upcoming events that are close to the given
				location and return those events to the application to be displayed. This is our most difficult challenge and
				one that will take a lot of testing. \\
					
				\item Design Rationale \\
				This viewpoint will require manipulating location data. Once the user's location has been determined, either
				through location services or through manual entry, we will have to perform a search through the database to
				find all events within a certain distance of that location. This will be the most difficult part of the entire
				task. Once the events have been found it will be relatively simple to show them to the user, sorted by
				distance, so that the user can view those closest to them easily. \\
			\end{enumerate}
			
		\item Composition Viewpoint \\
			\begin{enumerate}
				\item List tweets about events and/or people via the app and without \\
				Currently the prototype lists only official events found on meetups.com. It is important to also view events
				not listed on that website and that are posted via Twitter. Design concerns include:
				\begin{itemize}
					\item Designing a way to distinguish between a regular tweet, and a tweet with an event
					\item Linking twitter accounts with the right user and the matching account
					\item People who tweet about events, but are not a part of the application
					\item Search via API all about the meetup page and linked towards events
					\item Create entry in database about a certain person tweeting at a certain time
					\item Want to be able to see who tweeted, and able to re-tweet that tweet \\
				\end{itemize}
				
				\item Design View \\ %picture goes here!!!!!!!!!!!!!!!!!!!!!!!!!!!!!!!!!!!!!!!!!!!!!!!!!!!!!!!!!!!!!
				Our viewpoint for the tweeted events requirement is shown below with the communication between the
				application, Django framework, Twitter API, and the database. We plan to have the application
				communicate with the Twitter
				API such that the items that we find with the associated tags we will use will be displayed on the
				application and stored in the database. \\
				
				\item Design Rationale \\
				Once again the Twitter API will be useful in accomplishing a task. The API will be used to find tweets that
				have hashtags that are related to open source events. Once those have been acquired, they will be presented
				to users on the website as well as stored in the database for future reference. \\
			\end{enumerate}
	\end{enumerate}
\end{enumerate}

\subsection{Discussion on Design Document}
Blah blah blah discussion will go here.

%Tech review is inserted here.


\subsection{Discussion on Tech Review}
Blah blah blah discussion will go here.


\includepdf[pages={1}, pagecommand={\section{BLOG POSTS}}]{Blog_Posts.pdf}
\includepdf[pages={2-}, pagecommand={}]{Blog_Posts.pdf}


\includepdf[pages={-}, pagecommand={}, angle=90]{poster.pdf}

\section{PROJECT DOCUMENTATION}


\section{How did you learn the new technology?}
Throughout the presentation of our tools to our client Ross Gardler we were happy to hear that he was really excited about the progress we have made thus far.
In his excitement in seeing how far we have come with the tools and the progress he has seen
Ross began asking for feature requests to be added to the tools we have developed thus far.
We decided to add this section to our report so that we could reference Ross's confidence in our team to make these requests possible if we have time.
These requests are for future ideas for the tools as our client also would like us to submit patches for the tools
that we have made thus far as he approves of the work we have done. We will be completing these patches and getting them to Ross.
The requirements that have had feature requests proposed are shown below.

\section{What did you learn from all of this? (Justin Bruntmyer)}

\subsection{What Technical Information Did I Learn}

\section{What did you learn from all of this? (Megan Goossens)}
I learned quite a bit from this project. On the technical side, I learned a lot
about interfacing with APIs, and how documentation can either make them easy to
use, or an absolute pain to use. On the non-technical side, I learned about
writing even better reports, as well as how to explain our project in ways that
make it easy for people outside the open source community to understand.  In
project management I learned things like how frustrating it can be when your
client is hard to contact. If I had to do it all over again, I would work harder
to understand exactly what this project is about. We spent a lot of time trying
to understand what our client wanted from each requirement, time that would have
been better spent fulfilling them.

\section{What did you learn from all of this? (Hai Nguyen)}


\section{PROBLEMS ENCOUNTERED}

Throughout this term we have encountered many problems that have impeded our
progress for this project. This was to be expected as with all projects.  Events
occur that can halt progress and bring forth challenges that, as a group, we
needed to overcome. The problems that we have faced are listed below along with
the methods we used to get through the situation. We also ran into the occasioin
coding block where we would have days where not much progress was obtained.
However, we feel this did not impeded our progress as it is just part of being a
computer scientist.

\subsection{Problem 1}
Deciding which platform we were going to be developing on using Docker. This was
a major issue as we originally planed to work with the Windows operating system
however we could not get the application to run locally on a Windows platform.
We continually ran into errors with creating a local database along with having
the right Docker tools to support the application. We had available resources
such as a README file that gave insight on the problem but whatever we tried did
not seem to work. We eventually shifted gears and decided to try Docker and the
application on Linux, specifically Ubuntu 14.04. Thanks to the Linux knowledge
of Megan Goossens and plenty of online documentation we were able to get Docker
installed successfully along with running the application on a local host. From
this point we decided to continue developing in the Linux environment.

\subsection{Problem 2}
At the end of the Fall 2015 term we set up a weekly meeting with our client
throughout Winter 2016 to discuss implementation details for the week and
planned to utilize this time to make sure everyone is on the same page. Due to
some miss-communication we were unable to meet with our client for the first two
weeks. This halted our progress with because we had a lot of issues with getting
Docker to work with Windows and we were counting on a meeting with our client to
resolves those issues as soon as possible. We eventually got in contact with our
client and figured out what was happening as the first week our client was on an
unexpected trip to the UK and in the second week our client did not realize that
these meetings occurred every week throughout the term. These things happen and
once we all had a chance to get on the same page every weekly meeting is going
smoothly. This problem also happened at the beginning of the spring term. A new
weekly meeting time was set up with our client however our client believed that
our first meeting was in the middle of June. This was a problem as we tried to
set up a new meeting time but our client was very busy with travel and vacation
that we were not able to meet with him three weeks prior to expo. Email communication
was very difficult as well.

\subsection{Problem 3}
During week three of Winter 2016 we ran into the issue of meetings being
cancelled due to illness and injury. One of our team members experience an injury
that caused a full group meeting with a TA to occur which halted our progress in
having to get everyone caught up on the same page. This same week another group
member became sick and could not make it to two meetings for the week which
meant that two people could not make it so two meetings were cancelled out of the
weekly three meetings we have as a group. However, we were
able to work individually at home but it was still a noticeable disruption from
the normal work we produce in a week. It did not seem like it was going to
effect the group at first however when we began to get back on track it took
some adjustments to makes sure everyone was on the same page and try to make up
for the week we missed.

\subsection{Problem 4}
During the first week of the term we began developing based on the requirements
we had listed in our requirements document. The problem we ran into here was
that we had issues understanding what are requirements were trying to say thus
we went through the document and changed the language used for the requirements.
This did not change the requirement but it made it easier to understand if
someone is reading through it. This took a days worth of progress which was
frustrating due to the fact that we believed to have this done last term. We
currently are happy with the updated requirements document as it has been
approved by our client, professor, and TA. This halted our progress by being an
unnecessary step in the implementation process as it should have been completed
last term.

\end{document}
